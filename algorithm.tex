\chapter{Algorithm}


% Figure out how to add pic I made, do walkthrough to explain

% Talk about comm costs, differences from base

One of the advantages to our algorithm is that the runtime is heavily 
modifiable.  Some of our user parameters such as $m$ and $u$ are tuned according
to memory constraints or dataset size, but we can adjust $\tau$ and $\rho$ to 
bring down the runtime of our algorithm at the expense of accuracy, or increase 
the accuracy at the expense of the runtime.  Once you fit an initial model it is 
convenient to be able to optimize your accuracy, subject to your personal time 
constraints.  Consider our top tree, we know that when $\tau$ is 0 the model 
becomes metric, and that as $\tau$ increases the overlap will become greater and 
greater between pairs of children nodes.  It is therefore important to keep 
$\tau$ relatively low in the top tree, where a large amount of overlap will 
result in a greater compute time in the tope tree, as well as many more 
computations down the line when we are using our bottom trees.  In the bottom 
trees, however, it is less important to keep $\tau$ low, as multiple machines 
are making their spill tree computations in parallel, and we can afford to allow 
more overlap, as long as we set an appropriate value for $\rho$ to prevent the 
tree depth from getting out of hand.  Overall, the user defined parameters give 
our model a large degree of customization, and allow it to become optimized for 
the problem at hand.