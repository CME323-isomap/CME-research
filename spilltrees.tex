\chapter{Metric Trees and Spill Trees}

\vspace{5 mm}
\noindent
Please note that for this section we will be explaining how hybrid spill 
trees operate, so that we can then implement them in our final algorithm. 
As such, we will be paraphrasing and borrowing heavily from  \textit{``
An Investigation of Practical Approximate Nearest Neighbor Algorithms''},
which is cited below.

% SHOULD WE STILL CITE WITH [5] THROUGHOUT?

\vspace{5 mm}
\noindent
As mentioned earlier in the report, performing K nearest neighbor search for 
data geometry applications is computationally infeasible for large data sets. 
To alleviate this problem, we must approximate the solution. There are a 
variety of approximate solution techniques for K nearest neighbor search. We 
will focus on using hybrid spill trees and metric trees as our primary data 
structure.

\vspace{5 mm}
\noindent
Spill trees are specialized data storage constructs that are designed to 
efficiently organize a set of multidimensional points.  Spill trees operate by 
constructing a binary tree. The children in a full spill tree are a sub-set of t
he original data that empirically have the property of being close to one 
another. In other words, the spill tree is a way to look up data in which we 
can determine which data point is close to other observations.

\vspace{5 mm}
\noindent
Each parent node stores a classification method for whether a data point should 
go ``left'' or ``right'' along the node split. The classification method is 
done on whether the data point we wish to classify appears on either the left 
or right side of a boundary drawn through the data set (we'll talk about this 
boundary later in this report). The idea is that this boundary is constructed 
such that after a data point traverses the whole tree, it will have a high 
likelihood of ending up in a subset of data that is close to that point.

\vspace{5 mm}
\noindent
The method for constructing a spill tree is as follows:

\begin{itemize}
  \item Starting at the root of the tree, we consider our whole data set. At 
  each parent node, say $v$, we consider the subset of data that abide by the 
  decision bound created in the previous parent nodes.
  \item For each parent, we define two pivot nodes $p.lc$ and $p.rc$ such that 
  $p.lc$ and $p.rc$ have the maximum distance between them compared to all 
  other data points.
  \item We construct a hyperplane between $p.lc$ and $p.rc$ such that the line 
  between the two points is normal to the hyperplane and every point on the 
  hyperplane is equidistant from $p.lc$ and $p.rc$.
  \item Define $\tau$ as the overlap size, with $0 \le \tau < \infty$. $\tau$ 
  is a tuning parameter in our construction of the spill tree. We use $\tau$ 
  to construct two additional planes that are parallel to our initial 
  hyperplane and are distance $\tau$ away from this initial boundary hyperplane.
  \item We define the two new hyperplane boundaries as follows: the hyperplane 
  that is closest to $p.lc$ is $\tau.lc$ and the other closer to $p.rc$ is 
  $\tau.rc$.
  \item We finally split the data at node $v$ based on the rule that if the 
  data point is to the left of $\tau.rc$, we send it to the right child and if 
  it is to the right of $\tau.lc$ we send it to the left child. If a data point 
  satisfies both conditions, it is sent to both the right and left child.
  \item We keep splitting the data until the data size in each node hits a 
  specified threshold.
\end{itemize}
%CITE 2666

\vspace{5 mm}
\noindent
Notice that the use of the $\tau$ parameter allows for data to be duplicated if 
it appears within $\tau$ of the initial boundary split. The use of this 
parameter is the defining feature of a spill tree. Spill trees that set 
$\tau = 0$ are called metric trees.
%CITE 2666

\vspace{5 mm}
\noindent
The pseudo code for implementation is BLAH. The complexity for implementation is 
BLAH.

\vspace{5 mm}
\noindent
We would like to use a spill tree model on our data to categorize data with 
some likelihood of being close to our chosen data point. The re-sampling 
and duplication of data points inherent to spill trees (as apposed to metric 
trees) is ideal for a problem like K Nearest Neighbor. We frequently run into 
situations where data point $a$ might have neighbor data point $b$, but $b$ has 
other points that it is closer to (see figure). 
\bftext{INSERT FIGURE}
Without duplicates, $a$ would then be classified as a neighbor to $b$ and vice 
versa.
\bftext{ISTHIS TRUE?}

\vspace{5 mm}
\noindent
Further, if we wanted to use K Nearest Neighbor for classification, once we 
construct the spill tree, classifying a point would then be as simple as 
performing a look up for the point (which would have complexity $O(p d)$ where 
$d = $ the depth of the spill tree) and then computing K Nearest Neighbor on 
the subset of the data, which if we set a threshold of the size of our child 
node equal to $c$ would be $O(c^{2} p)$ for a total time complexity of 
$O(p d + c^{2} p)$. Compared to our naive way of doing K Nearest Neighbor, 
this classification scheme with $c << n$, our runtime on classification is much 
less.

\vspace{5 mm}
\noindent
However, there are numerous problems when implementing a spill tree directly 
on our whole data set.

\begin{itemize}
\item Finding the two pivots with the largest distance apart is already 
$O(n^{2} p)$, so implementing a spill tree may help us for lookup, but will not 
help us in avoiding the time complexity of actually fitting a geometric model.
\item For data size large enough that it will not fit on a single machine, 
calculating a spill tree directly in a distributed environment will be 
near-impossible. Like K Nearest Neighbor, to construct a spill tree with 
perfect precision, we need to compare each data point to every other data 
point to determine the furthest two data points. On a distributed setting, this 
is impossible without a lot of computer communication.
\end{itemize}

\vspace{5 mm}
\noindent
Instead of constructing a spill tree on all of our data, we will instead 
implement an approximate hybrid spill tree. This approximation method will 
construct a sampled hybrid spill tree on one machine. This spill tree will 
classify our data into chunks of data that can fit on single machines. From 
there, we construct a spill tree on each of the distributed data sets.

\vspace{5 mm}
\noindent
The main problem with using a normal spill tree as our data structure is that is
$\tau$ is too large, the depth of the spill tree can go to infinity.  To prevent
this, we can use a hybrid spill tree.  Hybrid spill trees operate identically to
spill trees, except they have an additional parameter $\rho$, with $0 \le \tau <
1$  such that if the overlap proportion when $P.lc$ and $P.rc$ are constructed
is  $> \tau$, then $P.lc$ and $P.rc$ are instead simply split on the initial
bound  $B$, with no overlap.  These nodes are marked as non-overlap, which will
become  important later on.  This guarantees that we will not have an infinite
depth  tree, and makes spill trees usable for our application.

\vspace{5 mm}
\noindent
We will be using a hybrid spill tree to store our data, because it will allow us
to make efficient approximate K nearest neighbor searches.  On a standard spill
tree with overlaps we would use defeatist search to implement K nearest
neighbors,  but it is important to remember that we now have non-overlap nodes
scattered  throughout, for which we would instead use the standard metric tree
depth first  search. \textbf{ADD CITATION TO 2666HEEEEEEEEEEEEEEEERRRRRRRRREEE}

\vspace{5 mm}
\noindent
This gives us an efficient data structure to use for our K nearest neighbors
algorithm, but it is still designed to be used in a serial environment, and we
want to use this in a distributed environment.  To accomplish this we will have
to somehow split the dataset among multiple machines.  This is problematic, as
to generate a hybrid spill tree, you need to have all of the data accessible in
memory.


