\chapter{Spill Trees}

\vspace{5 mm}
\noindent
When performing distributed K nearest neighbor search for geometry applications, 
the amount of computations invloved to compute an exact solution quickly 
becomes massive.  Especially for high-dimensional applications like isomaps, 
the computation costs are often prohibitively expensive.  To allieveate this 
problem, approximate solutions are often substituted for the less reasonable 
exact solution.  There are a variety of approximate solution techniques for K 
nearest neighbor search, but we will be using hybrid spill trees as our primary 
datastructure. 

\vspace{5 mm}
\noindent
Spill trees are specialized data storage constructs that are designed to 
efficiently organize a set of multidimensional points.  Spill trees operate by 
constructing a binary tree where each node of the tree represents a subset of 
the input points that were given to be stored.  At each node, starting with the 
node corresponding to the full dataset, we will construct two children nodes. 
These can be denoted v.lc and v.rc, in a spill tree these will be subsets of 
the parent node's corresponding points, and are allowed to overlap.  To 
construct these children nodes, the splitting procedure is as follows.

\begin{itemize}
  \item Define \tau as the overlap size, with $0 \le \tau \le \infty$
  \item Define $v$ as the parent node, and $v.lc$ and $v.rc$ as the children to be constructed
  \item Define $P$ as the parent node's point list, then $m$ is the midpoint of that list
  \item We now construct a decision boundary $B$ that goes through the midpoint $m$
  \item From this decision bound, we can construct two parallel decision bounds $LB$ and $RB$, both $\tau$ away from $B$
  \item Define $P.lc$ and $P.rc$ as the children nodes' point lists
  \item Then $P.lc = {x|x \in P, d(x,RB) +2\tau > d(x.LB)}$
  \item And $P.rc = {x|x \in P, d(x,LB) +2\tau > d(x.RB)}$
  \item Denote v.lc and v.rc as parent nodes and split on them, repeating until the point list size hts a certain threshold.
\end{itemize}
%CITE 2666

\vspace{5 mm}
\noindent
The main problem with using a normal spill tree as our datastructure is that is 
\tau is too large, the depth of the spill tree can go to infinity.  To prevent 
this, we can use a hybrid spill tree.  Hybrid spill trees operate identically to 
spill trees, except they have an additional parameter \rho, with $0 \le \tau < 1$ 
such that if the overlap proportion when P.lc and P.rc are constructed is 
$> \tau$, then $P.lc$ and $P.rc$ are instead simply split on the initial bound 
$B$, with no overlap.  These nodes are marked as non-overlap, which will become 
important later on.  This guarantees that we will not have an infinite depth 
tree, and makes spill trees usable for our application.

\vspace{5 mm}
\noindent
We will be using a hybrid spill tree to store our data, because it will allow us 
to make efficient approximate K nearest neighbor searches.  On a standard spill 
tree with overlaps we would use defeatist search to implement K nearest neighbors, 
but it is important to remember that we now have non-overlap nodes scattered 
throughout, for which we would instead use the standard metric tree depth first 
search. %ADD CITATION TO 2666 HEEEEEEEEEEEEEEEEEEEEEEEEEEEEEEEEEERRRRRRRRREEEEEE

\vspace{5 mm}
\noindent
This gives us an efficient datastructure to use for our K nearest neighbors algorithm, but it is still designed to be used in a serial environment, and we want to use this in a distributed environment.  To accomplish this we will have to somehow split the dataset among multiple machines.  This is problematic, as to generate a hybrid spill tree, you need to have all of the data accessible in memory.  


